\documentclass[11pt,a4paper]{article}
\usepackage{fontspec}
\usepackage{xunicode}
\usepackage{xeCJK}
\usepackage{geometry}
\geometry{left=2.5cm, right=2.5cm, top=2.5cm, bottom=2.5cm}

\usepackage{graphicx}
\usepackage{float}

\setmainfont{CMU Serif}
\setsansfont{CMU Serif}
\setmonofont{CMU Typewriter Text}

\setCJKmainfont{SimSun}
\setCJKsansfont{SimSun}
\setCJKmonofont{SimSun}

\usepackage{datetime}
\renewcommand{\today}{\number\year 年 \number\month 月 \number\day 日}
%\renewcommand{\abstractname}{}
%\renewcommand{\contentsname}{}

\title{``基于NI myDAQ的自动控制原理实验套件''设计说明}


\begin{document}
\author{电气信息学院 \ 自动控制原理实验室}

\maketitle

\begin{abstract}
  本项目以MATLAB/Simulink为实验环境,将National Instruments公司出品的myDAQ可编程测量设备作为模拟/数字接口,使计算机与外部硬件交互,进而设计了用于自动控制原理教学实验的三套实验设备。每个实验所需的硬件以一块板卡的形式与myDAQ连接,并可与MATLAB中的程序交互。通过更换板卡,学生可选择进行不同的实验。
\end{abstract}

\tableofcontents

\section{NI myDAQ 接线端子适配器}

\subsection{设计目的}
NI myDAQ可编程测量设备的右侧带有20个螺丝接线端子,原本被设计用于接入单根导线。考虑到项目中的实验板卡需要经常插拔,


\subsection{硬件选型与设计}

\subsection{原理图}

\section{模拟电路比例-积分-微分控制器}

\subsection{设计目的}

\section{直流电机调速实验}

\subsection{设计目的}

\section{温度控制实验}

\subsection{设计目的}


\end{document}
